\documentclass[10pt]{IEEEtran}

\usepackage{listings}
\usepackage[USenglish]{babel}
\usepackage{graphicx} % figuras
\usepackage{subfigure} % subfigura
\usepackage[utf8]{inputenc}
\usepackage{cite} 
\usepackage{hyperref}
\usepackage{algpseudocode}
%\usepackage[dvips]{graphicx}



\title{Tarea 2. Pendiente }

\author{Miguel Mendoza}
\newcounter{neq}
\begin{document}

\maketitle

\section{Introducci\'on}

\section{Desarrollo}

\section{Algoritmo}

\section{Resultados}


\section{Conclusion}


\section{Appendix}
A study was conducted at Virginia Tech to determine if certain static arm-strength measures have an influence on the “dynamic lift” characteristics of an individual. Twenty-five individuals were subjected to strength tests and then were asked to perform a weightlifting test in which weight was dynamically lifted overhead. The data are given here.\\

\begin{table}[htbp]
\begin{center}
\begin{tabular}{|l|l|l|}
\hline
Individual & Arm Strength x & Dynamic Lift y\\
\hline \hline 
1 & 17.3 & 71.7 \\ \hline
2 & 19.3 & 71.7 \\ \hline
3 & 19.5 & 88.3\\ \hline
4 & 19.7 & 75.0\\ \hline
5 & 22.9 & 91.7\\ \hline
6 & 23.1 & 100.0\\ \hline
7 & 26.4 & 73.3\\ \hline
8 & 26.8 & 65.0\\ \hline
9 & 27.6 & 75.0\\ \hline
10& 28.1&88.3\\ \hline
11& 28.2&68.3\\ \hline
12&28.7& 96.7\\ \hline
13&29.0&76.7\\ \hline
14&29.6&78.3\\ \hline
15&29.9&60.0\\ \hline
16&29.9&71.7\\ \hline
17&30.3&85.0\\ \hline
18&31.3&85.0\\ \hline
19&36.0&88.3\\ \hline
20&39.5&100.0\\ \hline
21&40.4&100.0\\ \hline
22&44.3&100.0\\ \hline
23&44.6&91.7\\ \hline
24&50.4&100.0\\ \hline
25&55.9&71.7\\ \hline
\end{tabular}
\caption{}
\label{tabla:sencilla}
\end{center}
\end{table}

Then: \\
(a) Estimate $\beta_{0}$ and $\beta_{1}$ for the linear regression curve $\mu_{{y}|{x}}$ = $\beta_{0} + \beta_{1} x$\\
(b) Find a point estimate of $\mu_{{y}|{30}}$.\\
(c) Plot the residuals versus the $x's$ (arm strength). Comment.\\

\textit{\textbf{Answer}}\\
(a) According with the formulas \ref{ec1} and \ref{ec2} $b_{0}$ and $b_{1}$ is calculated.

\begin{equation}
\label{ec1}
b_{1} = \frac{n\sum_{i=1}^{n}x_{i}y_{i}-(\sum_{i=1}^{n}x_{i})(\sum_{i=1}^{n}y_{i})}{n\sum_{i=1}^{n}x_{i}^{2} - (\sum_{i=1}^{n}x_{i})^{2}} 
\end{equation}

\begin{equation}
\label{ec2}
b_{0} = \frac{\sum_{i=1}^{n}y_{i} - b_{1}\sum_{i=1}^{n} x_{1}}{n}
\end{equation}
 Therefore,\\\\
$b_{1} = \frac{25 * 65164.04 - 778.7*2050}{25*26591.63 - 778.72} = 0.56$ and\\\\
$b_{0} = \frac{2050 - 0.56*778.7}{25} = 64.53$ .\\\\

Thus, estimating the linear regression:\\\\
$ \hat{y} = 64.53 + 0.56 * x$\\\\

(b) $\mu_{{y}|{30}} = 64.53 + 0.56 * (30) = 81.36$\\\\
(c) The ploted results are shown in the figure \ref{plot1}



\begin{figure}
\includegraphics[scale=0.5]{figure_1.png}
\caption{Residuals vs arm's strength}
\label{plot1}
\end{figure}


\bibliographystyle{acm}
\bibliography{tarea1}


\end{document}